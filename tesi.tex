%================Tesi di fine IV Anno TCMIA=============
%===================Leonardo Bentini====================
%Questo è codice sorgente da compilare solo con pdflatex.
%=======Editor consigliati: gedit, kile, vim.===========
%=======================================================

%+++++++++++++++Definizioni ed inclusioni+++++++++++++++
\documentclass[a4paper,14pt,twoside]{extarticle} %formato, tipo di documento, twoside alterna il margine in base alla parità delle pg.
\usepackage[utf8]{inputenc} %codifica
\usepackage[italian]{babel} %carica sillabazione e vocabolario italiani
\usepackage{extsizes}%permette dimensioni del testo maggiori, fino a 20pt
\usepackage{graphicx} %gestisce l'importazione di img
\usepackage[section,subsection,subsubsection]{inc/extraplaceins} %impedisce lo spostamento dell'img a fondo sezione
\usepackage{indentfirst}%rientro a prima riga
%\usepackage{degrade}%ridimensionatore di img
\usepackage[
		bookmarks=true,
		linktocpage=true, 
		colorlinks=false,
		pdfstartpage=1,
		bookmarksopen=1,
		pdfauthor={Leonardo Bentini},
		pdftitle={Tesina L. Bentini 4TCMIA},
		pdfkeywords={tesi, iv, 4, macchina, coster}
	]{hyperref}%hyperlink e informazioni sul file .pdf

\usepackage{fancyhdr}%headings
\frenchspacing %spaziatura ristretta
\makeindex %genera indice
\title{\sf \textbf{TESINA 5A ITI\\}}
\author{Bentini Leonardo \\ \\ \\ \\ V ASR\\ \\ \\ \\ \\ \\ \\ \\}
\date{Anno Scolastico 2014-2015}


\begin{document}
%\tableofcontents %genera indice

\begin{titlepage}%genera titolo
\setcounter{page}{0} %numera zero la pagina
\clearpage\maketitle
\thispagestyle{empty}
\newpage %pagina nuova

%pagina semivuota post-copertina
\newpage
\tiny.
\normalsize
\thispagestyle{empty}
\setcounter{page}{0}
%fine

\newpage
\clearpage\maketitle
\thispagestyle{empty}
\setcounter{page}{0}
\tableofcontents %genera indice
\end{titlepage}

%++++Gestione del nome sezione a cima pagina++++++
\setlength{\headheight}{15pt}
\pagestyle{fancy}
%\renewcommand{\sectionmark}[1]{\markright{#1}{} } %visualizza anche la subsezione
\renewcommand{\sectionmark}[1]{\markboth{#1}{}} % visualizza solo la sezione
\fancyhf{}
\fancyhead[RE,LO]{\textbf \thepage}%numero di pagina, posizione: Right Even Left Odd
\fancyhead[RO]{\textit{ \small{\nouppercase{\leftmark}}} }%nome sezione.
\fancyhead[LE]{\textit{ \small{ \nouppercase{\leftmark}}} }
\fancypagestyle{plain}{
\fancyhf{} % pulisce foglio
\renewcommand{\headrulewidth}{0pt} % rimuove le linee
\renewcommand{\footrulewidth}{0pt}
}
%\DegSetup{res=400,dir=/imgr,sdir=/redimg}
\newpage

%+++++++++++++Inizio corpo documento+++++++++++++
\section{Introduzione}

\newpage

\subsection{}

\subsubsection{}
%Immagine++++++++++++++++++++++++++++++++++++++++
\begin{figure}[ht!]
\sf
\centering
\includegraphics[scale=0.135,keepaspectratio=true]{./img/ass.jpg}
\caption{\small \sf \small ambiente tipicamente industriale del reparto di assemblaggio C2.}
\end{figure} 
%++++++++++++++++++++++++++++++++++++++++++++++++
\subsection{Permanenza su una macchina}
D
\subsection{Affiancamento in magazzino e turni}
In questo periodo, secondo disposizioni del tutor aziendale abbiamo svolto un affiancamento della durata di una settimana al personale del magazzino, facendoci un' idea di come vengono gestiti amministrativamente gli ordini\footnote{Lo stabilimento dispone di un ufficio dedito alla gestione della logistica e dei documenti di trasporto.}, le operazioni di carico/scarico sui camion e le operazioni del \emph{magazzino automatico} che conserva i pallet assegnandoli un codice a barre e una piazzola di carico; in attesa che il materiale venga richiamato all'esterno.

Ho poi svolto una settimana con orario di turno, dalle 6:00 alle 14:00, in affiancamento ad un singolo manutentore meccanico del reparto di assemblaggio, che ho seguito quindi durante tutta la sua giornata lavorativa.

Ci sono anche stati concessi alcuni giorni durante la terza settimana per completare il materiale necessario all'esame.
\newpage


\section{Analisi del rendimento}\label{sec:analisi}%Aggiunta del capitolo analisi della macchina, con tabelle ecc.
Il tutor ci ha affidato questo lavoro per tentare di \emph{migliorare la produttività} generale di un' assemblatrice, facendoci indagare direttamente sulle cause dello scarso rendimento, rilevato dalle statistiche di produzione.
\subsection{Fasi di lavoro}
Una volta ricevute le istruzioni, ci siamo organizzati per portare a termine il compito assegnatoci attenendoci ad alcune fasi:
\begin{itemize}

\item{\textbf{Studiare il funzionamento}, lo scopo e il metodo di produzione della macchina: 
nei primi giorni abbiamo seguito il percorso dei pezzi in lavorazione, verificando cosa avviene nelle varie zone del sistema.}
\item{\textbf{Localizzare i problemi} meccanici e controllare se essi sono causa di lavorazioni non portate a buon fine, creanti scarti.}
\item{\textbf{Dialogare con il personale} che si occupa del sistema, sia operatori che manutentori; cercando di scoprire i problemi ricorrenti o non risolvibili direttamente dall'azienda.}
\item{\textbf{Raccogliere i motivi di fermata} ad intervalli regolari, con l'ausilio di un computer portatile posizionato vicino alla macchina ed ovviamente di un blocco note.}
\item{\textbf{Trascrivere le statistiche} di produzione ed attività registrate automaticamente dal sistema informatico a bordo della macchina, come ad esempio il numero di pezzi orari teorici, i tempi di funzionamento e quelli di inattività.}
\item{\textbf{Elaborare i dati} raccolti e produrre un resoconto. (vedi sez.\ref{sec:dati}).}
%\item{\textbf{Impostare il resoconto} graficamente e strutturalmente.}
\end{itemize}

Una volta prodotto il resoconto lo abbiamo presentato al nostro tutor ed al direttore di stabilimento.
\newpage
\subsection{Problemi meccanici rilevati}\label{sec:prob}
Abbiamo riscontrato alcuni problemi di varia natura che abbassano la produttività dell' assemblatrice, aumentando il numero di scarti o il numero delle fermate.

Per esempio i più gravi sono stati:
\begin{itemize}
\item{Caduta del componente \emph{cerino} durate il moto della tavola D, causato da un problema di posizionamento dello stesso.}
\item{Frequente blocco delle \emph{molle} nel sistema di alimentazione, causato da deformità dei pezzi.}
\item{Il componente \emph{corpo} rimane spesso bloccato sulla pinza di prelievo, anche questo è dovuto alla deformità del pezzo.}
\item{Le \emph{guarnizioni interne} vengono posizionate a coppie sullo stesso pezzo in lavorazione, fermando il ciclo.}
\end{itemize}

%Immagine++++++++++++++++++++++++++++++++++++++++
\begin{figure}[ht!]
\centering \sf \tiny
\includegraphics[scale=0.26,keepaspectratio=true]{./img/cerini.jpg}
\caption{\small \sf tav. D, si possono vedere i cerini caduti.}
\end{figure}
%++++++++++++++++++++++++++++++++++++++++++++++++
%Queste informazioni sono state scritte ed allegate al resoconto delle analisi.
\newpage
\subsection{Resoconto analitico}\label{sec:dati}
Con i dati ottenuti sono stati sviluppati vari grafici e tabelle in formato elettronico.

Sono state impostate in totale tre tabelle, una per ogni settimana, contenenti il numero delle fermate della macchina, le principali cause, il numero di pezzi e il rapporto fra i tempi teorici di attività e quelli di lavoro utile, espresso in percentuale.

Per giustificare eventuali tempi di inattività elevati, è stato allegato il registro giornaliero, da noi creato, degli interventi di manutenzione, durante i quali la macchina deve rimanere ferma.

Abbiamo anche preparato un istogramma per ogni settimana che illustra il numero e i motivi delle fermate ricorrenti, in ordine decrescente.
%Immagine++++++++++++++++++++++++++++++++++++++++
\begin{figure}[ht!]
\centering \sf \tiny
\includegraphics[scale=0.75,keepaspectratio=true]{./img/graf.jpg}
\caption{\small \sf istogramma delle fermate.}
\end{figure}
%++++++++++++++++++++++++++++++++++++++++++++++++

In conclusione, posso dire che questo progetto è stato utile sia all'azienda, che ne ha tratto beneficio facendo risolvere al produttore dell'automazione i problemi segnalati, sia a noi allievi, in quanto potrebbe ricapitarci di dover diagnosticare ed esaminare un malfunzionamento, prima di un eventuale intervento.

\newpage
\section{Focus su un' automazione}\label{sec:focus}
Il sistema preso in esame è una macchina assemblatrice automatizzata di ultima generazione (fine 2012) con funzione di  controllo conformità pezzi integrata che ha lo scopo di produrre \emph{pompe nebulizzatrici} per l'erogazione di liquidi,  nel particolare prodotti di profumeria.

A pieno regime è in grado di assemblare 14400 pezzi/ora.\\
Costituita per la maggior parte di acciaio inossidabile, questa macchina è adatta anche ad essere utilizzata in ambienti con particolari standard di igiene.

È stata costruita dalla ditta italiana CB Automation, divisione della Bettinelli f.lli S.p.A.
%Immagine++++++++++++++++++++++++++++++++++++++++
\begin{figure}[ht!]
\tiny \sf
\centering
\includegraphics[scale=0.16,keepaspectratio=true]{./img/lato.jpg}
\caption{\small \sf la macchina vista di lato.}
\end{figure} 
%++++++++++++++++++++++++++++++++++++++++++++++++
\subsection{Meccanica}

L'assemblatrice opera caricando i componenti necessari e ponendoli su dei porta-pezzi mobilitati da delle "catene" ai quali sono saldati, queste sono chiamate \emph{transfer} e trasportano i pezzi da un punto in cui viene eseguita una lavorazione ad un altro.

I singoli punti sono chiamati \emph{stazioni}.
Ogni stazione ha un' unica funzione, come ad esempio il posizionamento e il fissaggio di un componente oppure un controllo di regolarità del pezzo in lavorazione.

Un \emph{gruppo} invece è un organo meccanico unico e mosso da una sola forza motrice, ma che può racchiudere varie stazioni.

%L'argomento verrà esplicato nella sezione \ref{sec:fasi}.

Una caratteristica importante della macchina che la differenzia da molte altre delle precedenti generazioni è l'uso di molti motori brushless comandati singolarmente dal PLC per mobilitare gli organi della macchina, in alternativa all'uso di un singolo albero a camme.

Per prelevare, trasferire ed operare sui pezzi, l'automazione utilizza un particolare tipo di organi di prelievo a vuoto, migliori sotto molti punti di vista rispetto alle pinze meccaniche.\footnote{Nella sezione \ref{sec:bracci} viene approfondito il funzionamento di questi apparati.}
Altra importante caratteristica è la presenza di vari punti di scarico scarti, atti ad eliminare questi prima delle altre lavorazioni, altrimenti inutili.

\subsection{Impianto di comando}
\
L'intero sistema è controllato da due \emph{Programmable Logical Controller} o \emph{PLC}: dei computer con la funzione di attivare delle connessioni elettriche in base alle informazioni che ricevono, comandando le elettrovalvole ed i motori elettrici riferendosi di solito allo stato meccanico della macchina rilevato da un \emph{encoder}: dispositivo che in base alla rotazione che compie il rotore al quale è collegato (ad esempio un albero a camme) pone in uscita un segnale elettronico analogico o digitale, generalmente interpretabile in gradi.

Ogni volta che il motore al quale è connesso l'encoder compie una rotazione completa, inizia un nuovo \emph{ciclo macchina}.

Nel caso in questione però l'encoder non è fisico, ma rimpiazzato da un temporizzatore software che svolge la stessa funzione ma con dei vantaggi: è possibile modificare molti parametri meccanici semplicemente dal monitor di controllo. (vedi sez. \ref{sec:soft}).

Tutti i dispositivi e le stazioni della macchina per funzionare correttamente devono effettuare l'operazione per la quale sono predisposti e ritornare in posizione di riposo entro la fine del ciclo macchina.

Se ciò non dovesse avvenire per qualche problema meccanico o di altra natura, esistono dei sensori atti a rilevare l'anomalia ed arrestare il ciclo.
Sarà poi necessaria una ricalibrazione automatica.
\subsubsection{Gestione e interfaccia}
L'assemblatrice dispone di una pulsantiera per il comando manuale e di otto punti di comando dai quali è possibile arrestare o riavviare il ciclo, oltre ad evocare un arresto di emergenza.

Sono presenti anche due monitor regolari e due touchscreen mediante i quali è possibile interagire con il software.
%Immagine++++++++++++++++++++++++++++++++++++++++
\begin{figure}[ht!]
\centering \sf
\includegraphics[scale=0.14,keepaspectratio=true]{./img/monitor.jpg}
\caption{\small \sf il pannello touchscreen, visibili le statistiche.}
\end{figure} 
%++++++++++++++++++++++++++++++++++++++++++++++++

\newpage
\subsubsection{Software}\label{sec:soft}

\subsection{Ringraziamenti}
\thanks{
Ringrazio i docenti del CFP e tutto il personale aziendale Coster, in particolare i manutentori al quale sono stato affiancato ed il tutor del periodo formativo in azienda.
}
\subsection{Bibliografia, sitografia e note}

Alcune informazioni riportate su questo fascicolo sono tratte da altre opere cartacee:
\begin{itemize}
\item{\textit{Manuale d'uso e manutenzione, CB Automation, 2012.}}
\item{\textit{Coster catalogue, Coster tecnologie speciali S.p.A., rev.25 2012.}}
\end{itemize}

Ho scelto di redarre questo documento utilizzando \emph{software libero}, principalmente col linguaggio di programmazione markup \large \LaTeX,
 \normalsize che assicura una buona resa tipografica.\\
Informazioni a riguardo disponibili su:

\begin{itemize}
\item{\url{http://it.wikipedia.org/wiki/Latex}}
%\item{\url{http://latex-project.org}}
\end{itemize}

\newpage
\thispagestyle{empty}
\footnotesize
	*******************\\
	Bentini Leonardo\\
	Tesina di fine quarto anno.\\
	Versione definitiva,\\
	ultima revisione: \today

\end{document}
